\chapter{Conclusioni}
Le misure qualitative dell'incidenza allo stallo e del numero di Strouhal sono soddisfacenti. \
Il campo di moto stimato presenta un errore medio elevato con valori massimi inacettabili anche solo per una stima qualitativa. La causa principale è da attribuirsi alla distorsione dell'immagine ed a un non perfetto allineamento della fotocamera [vedi fig \ref{fig: Confronto linee di corrente}].
Altri aspetti della prova sono migliorabili:
\begin{itemize}
\item l'esperienza acquisita durante l'elaborazione ha reso evidente che la pulizia del vetro è fondamentale: la prova è stata eseguita con un marcato alone di condensa e diverse particelle di polvere che hanno reso necessaria una correzione \emph{ad hoc}.
\item il profilo di materiale riflettente ha reso complicato identificare i fili di fumo nelle sue vicinanze, inconveniente evitabile utilizzando un profilo di colore scuro
\item la quantità di fumo immessa non è costante lungo il rastrello, alcuni fili tappati hanno ridotto la già di per sè bassa risoluzione spaziale del campo di moto
\item la luminosità diminuisce lungo l'altezza della galleria,ed in particolar modo sotto il profilo; un'illuminazione più uniforme faciliterebbe il tracciamento dei fili di fumo. Una luminosità più elevata consentirebbe inoltre un maggior rapporto segnale rumore a parità di tempo di esposizione. 
\end{itemize}

L'inesperienza ha inoltre influito negativamente sulla conduzione dell'esperimento: per compensare molti dei problemi affrontati sarebbe bastato prendere alcune accortezze:
\begin{itemize}
\item considerata la condizione di scarsa luce, utilizzare un obiettivo di apertura maggiore avrebbe consentito di acquisire maggior segnale a parità di tempo di esposizione
\item una griglia di calibrazione sarebbe stata utile per correggere in maniera più efficace la distorsione dell'obiettivo
\item acquisire alcuni secondi a galleria spenta prima di ogni prova avrebbe consentito di eliminare polvere,aloni ed il profilo stesso (come sperimentato nell'elaborazione dello shedding a 19 gradi)
\end{itemize} .\
